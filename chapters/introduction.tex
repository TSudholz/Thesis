\chapter*{\centering Introduction \\}\label{Ch:Intro}
\markright{Introduction}
\addcontentsline{toc}{chapter}{Introduction}
\pagenumbering{arabic}

%\begin{itemize}
%\item Define Cosmics Rays.
%\item The origins of the highest energy cosmic-rays still unknown.
%\item First detection by Pierre Auger in 1937 and the current detector looking at theses energies is the Pierre Auger Observatory.
%\item Hybrid experiment containing both surface detectors and fluorescence detectors
%\item Surface detector has nearly 100\% up-time while the fluorescence detectors only have 15\% up-time.
%\item **** Proposal to extend the fluorescence detector up-time. To achieve this will have to operator while the moon is above the horizon. This will increase the level NSB and will have the PMTs run under a reduced gain to compensate. ****
%\item Photomultiplier Tubes are used as pixels within the camera of the fluorescence detectors and  the aim of these thesis is to quantify the characteristics of the PMT under the reduced gain and increased.
%\item Outline a Summary of each chapter.
%\end{itemize}


Cosmic-rays are particles that originate outside of the Earth atmosphere. These particles can be photons, hadronic or leptonic in nature [ref?] and have been measurement over a large range of energies (over 6 decades in energy) since the time of their discovery. A spectrum has been built up over this energy range containing many interesting features with one of the longest running mysteries is what happens at the highest energy.  Since the first detection of extensive air showers by Pierre Auger in 1937 [ref], many different experiments have endeavoured to solve this mystery. The Pierre Auger Observatory [ref] is currently in operation to observe cosmic-rays at top end of the observable energy range.



%In this thesis, when mentioning cosmic-rays I will mean the hadronic component unless specified otherwise. Cosmic-rays have been measurement over a large range of energies (over 6 decades in energy) and it has many interesting features have been observed in this energy spectrum. One of the longest running mysteries is what happens at the highest energy. Since the first detection of extensive air showers by Pierre Auger in 1937 [ref], many different experiments have endeavoured to solve this mystery. The Pierre Auger Observatory [ref] is currently in operation to observe cosmic-rays at the highest energies. 

The Pierre Auger Observatory (PAO) is a hybrid experiment consisting of both surface detectors and fluorescence detectors. PAO is located at Malargue, Mendoza Province, Argentina. There are 1660 surface detectors within an area of 3000 km$^2$ with four fluorescence detector sites surrounding the surface detectors. The fluorescence detectors observe the light emitted from nitrogen molecules that have interacted with charged particles produced from the extensive air shower from the primary particle interacting in the atmosphere. The fluorescence telescopes can view a shower track that is within the Field of View. The surface detectors observe a shower when the secondary particles pass through a detector. The surface detectors only see particles that reach ground level with the fluorescence detectors can observe how the shower develops within the atmosphere. The advantages of the surface detectors is that they has a nearly 100\% operation up-time {ref}, while the fluorescence detectors only 15\% operation up-time [ref]. Also the fluorescence telescopes are used to calibrate the surface detectors.

A proposal in 2015 to extend the fluorescence detector operation up-time was entertained. Extending the FD up-time would be beneficial as the fluorescence detectors image the entire extensive air shower and would increase the number of showers observed through out yearly observation. Currently the FD telescopes gather data during sunless, moonless, cloudless and good weather periods. To achieve an extended operation time, the fluorescence detectors would have to operated while the moon is above the horizon. While the moon is up, this would increased the detected level of background noise (Night Sky Background) and increase the operation load of camera pixels of the Fluorescence detectors. The Fluorescence detectors using Photomultiplier tubes as camera pixels and PMT operations are sensitive to large changes in background noise. To compensate for the large increase in NSB, the PMT gain would be reduced. Current PMT operation is at the minimum of manufacturers specifications and it is wanted to know how the PMTs will act below this minimum specification.


- Need graph of expected variance in ADC$^2$ for the moon above the horizon for different phases.

- Want to increase the duty cycle of FD by measuring EAS under moonlight. Most likely observe under quarter to half moon. This will increased the NSB upto a factor of 10.

- The aim of increasing the duty cycle of FD is too measure more EAS at the highest energy band ($> 10^{19.5}$ eV).

- Need more statistics at highest energy band to complement SD measurements.

The aim of this thesis is to quantify the characteristics of the Photomultiplier Tubes operating under this reduced gain and outline any operation strategies. Outline of each chapter is as follows:

\begin{itemize}
\item  Chapter \ref{Ch:Cosmic-rays}: Cosmic-rays

Does this work as a new line

\item Chapter \ref{Ch:CR_Detection}: Detection of Cosmic-Rays

Add text here

\item Chapter \ref{Ch:PAO}: The Pierre Auger Observatory

Add text here

\item Chapter \ref{Ch:SelectEff} : Extensive Air Shower Selection Efficiency with Increased Night Sky Background 

Add text here

\item Chapter \ref{Ch:PMTCharacter} : Quantifying Characteristics of the Fluorescence Detector Photomultiplier Tubes 

Add text here

\item Chapter \ref{Ch:CompSimPMT} : Computer Simulation of the Fluorescence Detector Photomultiplier Tubes 

Add text here

\item Chapter \ref{Ch:GainVariance} : Measuring Gain Variance of the Fluorescence Detector Photomultiplier Tubes with CalA data 

Add text here

\item Chapter \ref{Ch:GainVariance_Results} : Results of Measuring the Gain Variance of the Fluorescence Detector Photomultiplier Tubes 

Add text here

\item Chapter \ref{Ch:LabPMTshift} : Laboratory Simulation of Fluorescence Detector Shifts

Add text here

\item Chapter \ref{Ch:CloudCuts} : Effectiveness of Cloud Camera Cuts on Data Set

Add text here

%\item Chapter \ref{Ch:Conclusion}: Conclusion 
%
%Future Work


\end{itemize}