\chapter{Research Proposal}

%\begin{enumerate}
%\item Upgrades
%\item extending the duty cycle of FD detectors
%\end{enumerate}

%Upgrades to the Pierre Auger Observatory are occurring. My part will involve investigating whether the Fluorescence Detectors (FD) can be operated under increased Night Sky Background (NSB) levels. This operation was proposed as there are very few of the highest energy showers detected by the FD and we would like to increase this number. Operating within increased NSB conditions would allow the FD to increase its duty cycle. The FD duty cycle is currently limited by requiring cloudless sky's, moon below the horizon, the sun to be at -18 \textdegree \ below the horizon and at least 3 hours of ideal conditions. The current FD duty cycle is 10 \% of the year. on
 
The Pierre Auger Observatory has been in operation for nearly 10 years and the collaboration is in the process of deciding on upgrades to extend and improve operation. My part will involve investigating whether the Fluorescence Detectors (FD) can be operated under increased Night Sky Background (NSB) levels to increase the FD duty cycle. The interest in increasing the FD duty cycle stems from the fact that there are very few detected events at the highest energy. Currently the FD is operated under these conditions: FD shifts are organised for nights with illuminated fraction of the moon less than 70\% and when the moon is below the horizon for longer than 3 hours. The telescope shutters are opened once the sun is at 18\textdegree \ below the horizon, if the average variance across the telescope camera is less than 100 ADC$^2$ (Analogue to Digital counts), and if individual pixels have variances less than 2000 ADC$^2$. This leads to a theoretical up-time of 22\% but after removal of short nights (less than 2 hours) and bad weather, the actual duty cycle of the FD is about 15\% of the year.

%Simulations done by other collaboration members have shown that the highest energy showers will still trigger the telescopes upto NSB levels equivalent of the full moon above the horizon. Currently there has been no investigation on the effect this will have on our instruments, mainly the photomultiplier tubes (PMTs). Full moon is probably a little bit too ambitious, where the main increased NSB levels would be around quarter moon. If the FD is operated, the duty cycle can be double to 20 \%. The increased NSB level would be a factor of 10.

To extend the FD duty cycle we propose to relax the conditions under which the FD are operated. The conditions that have the possibility of being relaxed are the threshold on the variance across the FD cameras and where the moon is in the night sky \cite{GAP1994-034,GAP2011-039}. Relaxing these conditions would theoretically double the FD duty cycle but would mean operating under a factor of 10 increased NSB levels.  Simulations done by other collaboration members have shown that the highest energy showers will still trigger the telescopes and be reconstructed successfully under up to 40 times the typical measured NSB levels (this is equivalent to the full moon above the horizon). Work has been started on the effects that the increased NSB levels have on the equipment, mainly the photomultiplier tubes (PMTs). In the past, it was shown that the PMT operation was affected when exposed to higher levels of NSB.

% upto NSB levels equivalent of the full moon above the horizon

%To counter the increased NSB level, it is proposed that the PMTs are operated under decreased gain. Therefore PMT characteristics are neead to be investigated. The PMT characteristics to be investigated are linearity, dynamic range, and operation when changed from different gain levels.

To counter the increased NSB level, it is proposed that the PMTs could be operated under decreased gain. Investigation of the PMTs characteristics are needed. The characteristics needed to be quantified are linearity, ageing, and operation when gain is changed from one level to another. Determining a PMT region of linearity is important as it shows the region where if the number of incoming photons are doubled, the signal out of PMT is doubled. How a PMT ages is an important characteristic. A PMT operation life span is quoted in terms of total charge ccollected at the anode. It is expected that under the increased NSB levels that the PMTs will age more rapidly. This is the main reason for proposing operating the PMT under a lower gain to reduce anode current. The characteristics of  operating the PMT at lower gain needs to quantified so it can be compared to operation at the typical gain level. 
