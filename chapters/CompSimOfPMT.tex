\chapter[Computer Simulation of FD PMT]{\centering Computer Simulation of Fluorescence Detector Photomultiplier Tubes \\}\label{Ch:CompSimPMT}

Simulating the FD PMT under differing NSB and for different reasons.
\begin{itemize}
\item Theoretical value for Gain Variance
\item PMT Gain Variance
\item Show both for flat distribution and Gaussian variations for dynodes
\item Results
\item FD FLT under increased NSB
\end{itemize}

\section{Motivation}

- PMT Theory/Background

\section{Method and Theory}

For a deeper understanding of how the PMT gain variance works I used a toy model Monte Carlo to  simulate electrons moving throughout the PMT dynode chain. Gain Variance is the Gaussian broadening of the measured anode signal. The number of electrons that are emitted from each dynode follows a Poisson distribution.  When an electron hits a dynode the number of electrons emitted follow a probability distribution that is Poission in shape. As electrons travel further down the dynode chain and the number of electrons emitted per dynode increase the distribution becomes more Gaussian like.

To simulate the Gain Variance within a PMT I set-up my toy model to match the number of stages with the XP3062 PMT. I was only looking at the single photo-electron case (ie. where only a single electron is emitted from the cathode.). If there was no extra broadening then the distribution would be dictated by the Poisson distribution. For each electron that hits a dynode a random number generator is used to determine the number of electrons that leave the dynode.

- Lot more detail. Through how the toy model is produced.

\textbf{Draw flow chart of toy model Monte Carlo.}

\subsection{Theoretical value of Gain Variance?}

I investigated different scenarios - firstly each dynode able to emit electrons with uniform Poission distribution across its surface and secondly each dynode having some sort of irregularity across its surface. The irregularity was added by shifting the mean of the Poisson distribution by sampling from a random Gaussian distribution. The gain variance was investigated at different PMT gain values. The PMT gain values picked was equivalent to a high voltage across the PMT of 1300V, 900V and 600V. The high voltage of 1300V was picked as a direct gain variance measurement was preformed by  \textbf{some group within Auger}, 900V is the typical high voltage used when the FD telescopes are operating and 600V is the approximate high voltage that is expected to be used when the FD telescopes are observing under increased NSB conditions. 

\begin{figure}
\centering
\begin{subfigure}[b]{0.44\textwidth}
\adjincludegraphics[width=\textwidth, trim={0 0 0 {0.08\height}}, clip]{chapters/graphs/PMTsimulation/ElectronsLeaving_PoissonFit_Dynode1_5e4.pdf}
\caption{Distribution of simulated electrons leaving dynode 1. The red line is a fitted Poisson distribution.}
\end{subfigure}
\hspace{3mm}
\begin{subfigure}[b]{0.44\textwidth}
\adjincludegraphics[width=\textwidth, trim={0 0 0 {0.08\height}}, clip]{chapters/graphs/PMTsimulation/ElectronsLeaving_Dynode2_5e4.pdf}
\caption{Distribution of simulated electrons leaving dynode 2. The red line is a fitted Gaussian distribution.}
\end{subfigure}

\vspace{3mm}

\begin{subfigure}[b]{0.44\textwidth}
\adjincludegraphics[width=\textwidth, trim={0 0 0 {0.08\height}}, clip]{chapters/graphs/PMTsimulation/ElectronsLeaving_Dynode3_5e4.pdf}
\caption{Distribution of simulated electrons leaving dynode 3. The red line is a fitted Poisson distribution.}
\end{subfigure}
\hspace{3mm}
\begin{subfigure}[b]{0.44\textwidth}
\adjincludegraphics[width=\textwidth, trim={0 0 0 {0.08\height}}, clip]{chapters/graphs/PMTsimulation/ElectronsLeaving_Dynode4_5e4.pdf}
\caption{Distribution of simulated electrons leaving dynode 4. The red line is a fitted Gaussian distribution.}
\end{subfigure}

\vspace{3mm}

\begin{subfigure}[b]{0.44\textwidth}
\adjincludegraphics[width=\textwidth, trim={0 0 0 {0.08\height}}, clip]{chapters/graphs/PMTsimulation/ElectronsLeaving_Dynode5_5e4.pdf}
\caption{Distribution of simulated electrons leaving dynode 5. The red line is a fitted Poisson distribution.}
\end{subfigure}
\hspace{3mm}
\begin{subfigure}[b]{0.44\textwidth}
\adjincludegraphics[width=\textwidth, trim={0 0 0 {0.08\height}}, clip]{chapters/graphs/PMTsimulation/ElectronsLeaving_Dynode6_5e4.pdf}
\caption{Distribution of simulated electrons leaving dynode 6. The red line is a fitted Gaussian distribution.}
\end{subfigure}

\vspace{3mm}

\begin{subfigure}[b]{0.44\textwidth}
\adjincludegraphics[width=\textwidth, trim={0 0 0 {0.08\height}}, clip]{chapters/graphs/PMTsimulation/ElectronsLeaving_Dynode7_5e4.pdf}
\caption{Distribution of simulated electrons leaving dynode 7. The red line is a fitted Poisson distribution.}
\end{subfigure}
\hspace{3mm}
\begin{subfigure}[b]{0.44\textwidth}
\adjincludegraphics[width=\textwidth, trim={0 0 0 {0.08\height}}, clip]{chapters/graphs/PMTsimulation/ElectronsLeaving_Dynode8_5e4.pdf}
\caption{Distribution of simulated electrons leaving dynode 8. The red line is a fitted Gaussian distribution.}
\end{subfigure}
\end{figure}

\section{Results of PMT Gain Variance Simulation}


\begin{figure}
\centering
\begin{subfigure}[b]{0.44\textwidth}
\adjincludegraphics[width=\textwidth, trim={0 0 0 {0.08\height}}, clip]{chapters/graphs/PMTsimulation/PMTsim_1PE_TypeC_DynodeSigm0_5e5.pdf}
\caption{Simulated observed signal from PMT anode with 8 stages and gain of 5 $\times \ 10^5$. Poisson fluctuations at each stage only. No each Gaussian broadening at any dynode.}
\end{subfigure}
\hspace{3mm}
\begin{subfigure}[b]{0.44\textwidth}
\adjincludegraphics[width=\textwidth, trim={0 0 0 {0.08\height}}, clip]{chapters/graphs/PMTsimulation/PMTsim_1PE_TypeC_DynodeSigm0_1_5e5.pdf}
\caption{Simulated observed signal from PMT anode with 8 stages and gain of 5 $\times \ 10^5$. Poisson fluctuations at each stage only. Added Gaussian braodening at each dynode of 10\%.}
\end{subfigure}

\vspace{3mm}

\begin{subfigure}[b]{0.44\textwidth}
\adjincludegraphics[width=\textwidth, trim={0 0 0 {0.08\height}}, clip]{chapters/graphs/PMTsimulation/PMTsim_1PE_TypeC_DynodeSigm0_2_5e5.pdf}
\caption{Simulated observed signal from PMT anode with 8 stages and gain of 5 $\times \ 10^5$. Poisson fluctuations at each stage only. Added Gaussian braodening at each dynode of 20\%.}
\end{subfigure}
\hspace{3mm}
\begin{subfigure}[b]{0.44\textwidth}
\adjincludegraphics[width=\textwidth, trim={0 0 0 {0.08\height}}, clip]{chapters/graphs/PMTsimulation/PMTsim_1PE_TypeC_DynodeSigm0_3_5e5.pdf}
\caption{Simulated observed signal from PMT anode with 8 stages and gain of 5 $\times \ 10^5$. Poisson fluctuations at each stage only. Added Gaussian braodening at each dynode of 30\%.}
\end{subfigure}

\vspace{3mm}

\begin{subfigure}[b]{0.44\textwidth}
\adjincludegraphics[width=\textwidth, trim={0 0 0 {0.08\height}}, clip]{chapters/graphs/PMTsimulation/PMTsim_1PE_TypeC_DynodeSigm0_4_5e5.pdf}
\caption{Simulated observed signal from PMT anode with 8 stages and gain of 5 $\times \ 10^5$. Poisson fluctuations at each stage only. Added Gaussian braodening at each dynode of 40\%.}
\end{subfigure}
\hspace{3mm}
\begin{subfigure}[b]{0.44\textwidth}
\adjincludegraphics[width=\textwidth, trim={0 0 0 {0.08\height}}, clip]{chapters/graphs/PMTsimulation/PMTsim_1PE_TypeC_DynodeSigm0_5_5e5.pdf}
\caption{Simulated observed signal from PMT anode with 8 stages and gain of 5 $\times \ 10^5$. Poisson fluctuations at each stage only. Added Gaussian braodening at each dynode of 50\%.}
\end{subfigure}
\end{figure}

\begin{figure}
\centering
\begin{subfigure}[b]{0.44\textwidth}
\adjincludegraphics[width=\textwidth, trim={0 0 0 {0.08\height}}, clip]{chapters/graphs/PMTsimulation/PMTsim_1PE_TypeC_DynodeSigm0_5e4.pdf}
\caption{Simulated observed signal from PMT anode with 8 stages and gain of 5 $\times \ 10^4$. Poisson fluctuations at each stage only. No each Gaussian broadening at any dynode.}
\end{subfigure}
\hspace{3mm}
\begin{subfigure}[b]{0.44\textwidth}
\adjincludegraphics[width=\textwidth, trim={0 0 0 {0.08\height}}, clip]{chapters/graphs/PMTsimulation/PMTsim_1PE_TypeC_DynodeSigm0_1_5e4.pdf}
\caption{Simulated observed signal from PMT anode with 8 stages and gain of 5 $\times \ 10^4$. Poisson fluctuations at each stage only. Added Gaussian braodening at each dynode of 10\%.}
\end{subfigure}

\vspace{3mm}

\begin{subfigure}[b]{0.44\textwidth}
\adjincludegraphics[width=\textwidth, trim={0 0 0 {0.08\height}}, clip]{chapters/graphs/PMTsimulation/PMTsim_1PE_TypeC_DynodeSigm0_2_5e4.pdf}
\caption{Simulated observed signal from PMT anode with 8 stages and gain of 5 $\times \ 10^4$. Poisson fluctuations at each stage only. Added Gaussian braodening at each dynode of 20\%.}
\end{subfigure}
\hspace{3mm}
\begin{subfigure}[b]{0.44\textwidth}
\adjincludegraphics[width=\textwidth, trim={0 0 0 {0.08\height}}, clip]{chapters/graphs/PMTsimulation/PMTsim_1PE_TypeC_DynodeSigm0_3_5e4.pdf}
\caption{Simulated observed signal from PMT anode with 8 stages and gain of 5 $\times \ 10^4$. Poisson fluctuations at each stage only. Added Gaussian braodening at each dynode of 30\%.}
\end{subfigure}

\vspace{3mm}

\begin{subfigure}[b]{0.44\textwidth}
\adjincludegraphics[width=\textwidth, trim={0 0 0 {0.08\height}}, clip]{chapters/graphs/PMTsimulation/PMTsim_1PE_TypeC_DynodeSigm0_4_5e4.pdf}
\caption{Simulated observed signal from PMT anode with 8 stages and gain of 5 $\times \ 10^4$. Poisson fluctuations at each stage only. Added Gaussian braodening at each dynode of 40\%.}
\end{subfigure}
\hspace{3mm}
\begin{subfigure}[b]{0.44\textwidth}
\adjincludegraphics[width=\textwidth, trim={0 0 0 {0.08\height}}, clip]{chapters/graphs/PMTsimulation/PMTsim_1PE_TypeC_DynodeSigm0_5_5e4.pdf}
\caption{Simulated observed signal from PMT anode with 8 stages and gain of 5 $\times \ 10^4$. Poisson fluctuations at each stage only. Added Gaussian braodening at each dynode of 50\%.}
\end{subfigure}
\end{figure}

\begin{figure}
\centering
\begin{subfigure}[b]{0.44\textwidth}
\adjincludegraphics[width=\textwidth, trim={0 0 0 {0.08\height}}, clip]{chapters/graphs/PMTsimulation/PMTsim_1PE_TypeC_DynodeSigm0_5e3.pdf}
\caption{Simulated observed signal from PMT anode with 8 stages and gain of 5 $\times \ 10^3$. Poisson fluctuations at each stage only. No each Gaussian broadening at any dynode.}
\end{subfigure}
\hspace{3mm}
\begin{subfigure}[b]{0.44\textwidth}
\adjincludegraphics[width=\textwidth, trim={0 0 0 {0.08\height}}, clip]{chapters/graphs/PMTsimulation/PMTsim_1PE_TypeC_DynodeSigm0_1_5e3.pdf}
\caption{Simulated observed signal from PMT anode with 8 stages and gain of 5 $\times \ 10^3$. Poisson fluctuations at each stage only. Added Gaussian braodening at each dynode of 10\%.}
\end{subfigure}

\vspace{3mm}

\begin{subfigure}[b]{0.44\textwidth}
\adjincludegraphics[width=\textwidth, trim={0 0 0 {0.08\height}}, clip]{chapters/graphs/PMTsimulation/PMTsim_1PE_TypeC_DynodeSigm0_2_5e3.pdf}
\caption{Simulated observed signal from PMT anode with 8 stages and gain of 5 $\times \ 10^3$. Poisson fluctuations at each stage only. Added Gaussian braodening at each dynode of 20\%.}
\end{subfigure}
\hspace{3mm}
\begin{subfigure}[b]{0.44\textwidth}
\adjincludegraphics[width=\textwidth, trim={0 0 0 {0.08\height}}, clip]{chapters/graphs/PMTsimulation/PMTsim_1PE_TypeC_DynodeSigm0_3_5e3.pdf}
\caption{Simulated observed signal from PMT anode with 8 stages and gain of 5 $\times \ 10^3$. Poisson fluctuations at each stage only. Added Gaussian braodening at each dynode of 30\%.}
\end{subfigure}

\vspace{3mm}

\begin{subfigure}[b]{0.44\textwidth}
\adjincludegraphics[width=\textwidth, trim={0 0 0 {0.08\height}}, clip]{chapters/graphs/PMTsimulation/PMTsim_1PE_TypeC_DynodeSigm0_4_5e3.pdf}
\caption{Simulated observed signal from PMT anode with 8 stages and gain of 5 $\times \ 10^3$. Poisson fluctuations at each stage only. Added Gaussian braodening at each dynode of 40\%.}
\end{subfigure}
\hspace{3mm}
\begin{subfigure}[b]{0.44\textwidth}
\adjincludegraphics[width=\textwidth, trim={0 0 0 {0.08\height}}, clip]{chapters/graphs/PMTsimulation/PMTsim_1PE_TypeC_DynodeSigm0_5_5e3.pdf}
\caption{Simulated observed signal from PMT anode with 8 stages and gain of 5 $\times \ 10^3$. Poisson fluctuations at each stage only. Added Gaussian braodening at each dynode of 50\%.}
\end{subfigure}
\end{figure}

\section{Simulation of Gain Variance Method}

\begin{itemize}
\item Show off histogram of simulated Gain Variance Ratios
\item Show that there's a natural spread
\item some of the calculated ratio's will be less then one due to spread in measure variances
\item differences in methods? Pairs vs averages method should return the same results
\item benefits of combining noise traces over using just 140 bins per trace.
\end{itemize}

\section{FD FLT under different NSB levels}

Looked at study to simulate the effects of increased NSB on the First Level Trigger (FLT). The Auger maintains the FLT to trigger at 100 Hz range. From simulations with different NSB levels it can be seen trigger threshold above the mean would be required to maintain the expected trigger level. The NSB of 2.71$\backslash$100 ns is the typical NSB level observed at the site. From there I picked 5 increased NSB levels to show how threshold above the mean changes. 


\begin{figure}
\centering
\begin{subfigure}[b]{0.44\textwidth}
\adjincludegraphics[width=\textwidth, trim={0 0 0 {0.08\height}}, clip]{chapters/graphs/PMTsimulation/FLT_simulation_2_71pePER100ns.pdf}
\caption{FLT simulation with NSB of 2.71 pe / 100ns.}
\end{subfigure}
\hspace{3mm}
\begin{subfigure}[b]{0.44\textwidth}
\adjincludegraphics[width=\textwidth, trim={0 0 0 {0.08\height}}, clip]{chapters/graphs/PMTsimulation/FLT_simulation_6_60pePER100ns.pdf}
\caption{FLT simulation with NSB of 6.60 pe / 100ns.}
\end{subfigure}

\vspace{3mm}

\begin{subfigure}[b]{0.44\textwidth}
\adjincludegraphics[width=\textwidth, trim={0 0 0 {0.08\height}}, clip]{chapters/graphs/PMTsimulation/FLT_simulation_46_8pePER100ns.pdf}
\caption{FLT simulation with NSB of 46.8 pe / 100ns.}
\end{subfigure}
\hspace{3mm}
\begin{subfigure}[b]{0.44\textwidth}
\adjincludegraphics[width=\textwidth, trim={0 0 0 {0.08\height}}, clip]{chapters/graphs/PMTsimulation/FLT_simulation_65_8pePER100ns.pdf}
\caption{FLT simulation with NSB of 65.8 pe / 100ns.}
\end{subfigure}

\vspace{3mm}
\begin{subfigure}[b]{0.44\textwidth}
\adjincludegraphics[width=\textwidth, trim={0 0 0 {0.08\height}}, clip]{chapters/graphs/PMTsimulation/FLT_simulation_263pePER100ns.pdf}
\caption{FLT simulation with NSB of 263 pe / 100ns.}
\end{subfigure}
\hspace{3mm}
\begin{subfigure}[b]{0.44\textwidth}
\adjincludegraphics[width=\textwidth, trim={0 0 0 {0.08\height}}, clip]{chapters/graphs/PMTsimulation/FLT_simulation_657pePER100ns.pdf}
\caption{FLT simulation with NSB of 657 pe / 100ns.}
\end{subfigure}
\end{figure}