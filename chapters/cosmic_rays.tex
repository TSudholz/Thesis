\chapter[Cosmic-Rays]{\centering Cosmic-Rays \\}\label{Ch:Cosmic-rays}

\section{History of Cosmic-Rays}

First detection of ionizing radiation. 

1785: Coulomb found that 
electroscopes can spontaneously 
discharge by the action of the air 
and not by defective insulation

1835: Faraday confirms the 
observation by Coulomb, with 
better insulation technology

1879: Crookes measures that the 
speed of discharge of an 
electroscope decreased when 
pressure was reduced 

\section{Energy Spectrum and Mass composition}

\begin{figure}[hp]
\centering
\includegraphics[width=\textwidth]{chapters/pix/CosmicRay_Spectrum.png}
\caption{Measured energy spectrum of cosmic-rays from 100 GeV up to the highest detected energy.}
\label{fig:CR_Spectrum}
\end{figure}

Cosmic-rays have been detected over a large range of energies from GeV (10$^9$ \ eV) to above EeV (10$^18$ \ eV). Spectrum in Figure \ref{fig:CR_Spectrum} shows the break at the knee and ankle and which type of experiments are most suited to measurement each part. Cosmic-ray spectrum starts out at E$^{-2}$ \ and can be as steep as E$^{-2.7}$ at the highest energies.

Cosmic-rays can consist of protons to iron. 

CR spectrum has many feature. Main features are the knee, second knee and ankle. The knee is around 3 $\times$ 10$^{15}$.

Pierre Auger Observatory measurement of isotropy that shows that the cosmic-ray spectrum changes from predominately galactic to extra-galactic at the ankle.

Predicted Greisen-Zatsepin-Kuz’min (GZK) cut-off about 6 $\times$ 10$^{19}$. Cosmic-rays above this energy are theorised to interact with the cosmic microwave background radiation. Greisen independently of Kuz'man and Zatsepin all predicted this energy loss.

Pierre Auger Observatory measurement of Xmax and the second moment $\sigma$(Xmax) has mass composition information as well how this changes as a function of energy.

\section{Production Method and Sources}

- Bottom-Up Acceleration 

Supernova explosions 

AGN jets

other energetic processes

dark matter annihilations.

- Top-Down Acceleration