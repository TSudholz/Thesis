\chapter{Detections of Cosmic-Rays}\label{Ch:CR_Detection}

\section{Extensive Air Showers}

Use Earth's atmosphere as an interaction medium.
Primary particle interacts with the molecules in the atmosphere to produce a cascade of secondary particles. This cascade of particles is referred to as an Extensive Air Shower (EAS).
Hadronic primaries can produce pions, muons and other stuff.
Mixture of a hadronic core with an electromagnetic component from the decay of $\pi^{0}$.

Shower profile has particles produced until energy on individual secondary particles drop below the ionization threshold. Therefore the shower will reach a point of maximum particle number then will drop off.

\section{Fluorescence Production}
The charge particles of EAS interact with the nitrogen molecules in the atmosphere. This interaction turns the nitrogen molecule dipole like and when the nitrogen returns to a ground state, a photon is emitted. This emitted photon is termed fluorescence light. Fluorescence light is can be emitted isotropically and typically in the UV band (between 300 and 400 nm). *** Show wavelength profile ***


\section{Atmospheric Effects}



\section{Detectors and History}