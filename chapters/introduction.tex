\chapter*{\centering Introduction \\}\label{Ch:Intro}
\markright{Introduction}
\addcontentsline{toc}{chapter}{Introduction}
\pagenumbering{arabic}

\begin{itemize}
\item Define Cosmics Rays.
\item The origins of the highest energy cosmic-rays still unknown.
\item First detection by Pierre Auger in 1937 and the current detector looking at theses energies is the Pierre Auger Observatory.
\item Hybrid experiment containing both surface detectors and fluorescence detectors
\item Surface detector has nearly 100\% up-time while the fluorescence detectors only have 15\% up-time.
\item **** Proposal to extend the fluorescence detector up-time. To achieve this will have to operator while the moon is above the horizon. This will increase the level NSB and will have the PMTs run under a reduced gain to compensate. ****
\item Photomultiplier Tubes are used as pixels within the camera of the fluorescence detectors and  the aim of these thesis is to quantify the characteristics of the PMT under the reduced gain and increased.
\item Outline a Summary of each chapter.
\end{itemize}


Cosmic-rays are particles that originate outside of the Earth atmosphere. These particles can be photons, hadronic or leptonic in nature [ref?]. In this thesis, when mentioning cosmic-rays I will mean the hadronic component unless specified otherwise. Cosmic-rays have been measurement over a large range of energies (over 6 decades in energy) and it has many interesting features have been observed in this energy spectrum. One of the longest running mysteries is what happens at the highest energy. Since the first detection of extensive air showers by Pierre Auger in 1937 [ref], many different experiments have endeavoured to solve this mystery. The Pierre Auger Observatory [ref] is currently in operation to observe cosmic-rays at the highest energies. 

The Pierre Auger Observatory is a hybrid experiment consisting of both surface detectors and fluorescence detectors. (Outline location) The surface detector has a nearly 100\% operation up-time {ref} while the fluorescence detectors only 15\% operation up-time [ref]. (Outline how Auger detects cosmic-rays, just need a brief summary).

A current proposal to extend the fluorescence detector operation up-time. Extended up-time would be beneficial as the fluorescence detectors image the entire extensive air shower and would increase the number of showers observed through out yearly observation. To achieve the extended operation the fluorescence detectors would have to operated while the moon is above the horizon. While the moon is up, this would increased the Night Sky Background level and to compensate the Photomultiplier Tubes acting as the camera pixels would have to be run under reduced gain. 

- Need graph of expected variance in ADC$^2$ for the moon above the horizon for different phases.

- Want to increase the duty cycle of FD by measuring EAS under moonlight. Most likely observe under quarter to half moon. This will increased the NSB upto a factor of 10.

- The aim of increasing the duty cycle of FD is too measure more EAS at the highest energy band ($> 10^{19.5}$ eV).

- Need more statistics at highest energy band to complement SD measurements.

The aim of this thesis is to quantify the characteristics of the Photomultiplier Tubes operating under this reduced gain and outline any operation strategies. Outline of each chapter is as follows:

\begin{itemize}
\item  Chapter \ref{Ch:Cosmic-rays}: Cosmic-rays

Does this work as a new line

\item Chapter \ref{Ch:CR_Detection}: Detection of Cosmic-Rays

Add text here

\item Chapter \ref{Ch:PAO}: The Pierre Auger Observatory

Add text here

\item Chapter \ref{Ch:SelectEff} : Extensive Air Shower Selection Efficiency with Increased Night Sky Background 

Add text here

\item Chapter \ref{Ch:PMTCharacter} : Quantifying Characteristics of the Fluorescence Detector Photomultiplier Tubes 

Add text here

\item Chapter \ref{Ch:CompSimPMT} : Computer Simulation of the Fluorescence Detector Photomultiplier Tubes 

Add text here

\item Chapter \ref{Ch:GainVariance} : Measuring Gain Variance of the Fluorescence Detector Photomultiplier Tubes with CalA data 

Add text here

\item Chapter \ref{Ch:GainVariance_Results} : Results of Measuring the Gain Variance of the Fluorescence Detector Photomultiplier Tubes 

Add text here

\item Chapter \ref{Ch:LabPMTshift} : Laboratory Simulation of Fluorescence Detector Shifts

Add text here

\item Chapter \ref{Ch:CloudCuts} : Effectiveness of Cloud Camera Cuts on Data Set

Add text here

%\item Chapter \ref{Ch:Conclusion}: Conclusion 
%
%Future Work


\end{itemize}